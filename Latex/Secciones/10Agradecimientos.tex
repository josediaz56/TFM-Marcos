En la realización de este trabajo he recibido ayuda de diferentes personas y me gustaría agradecer a todos estos el haber utilizado parte de su tiempo en mi.

En primer lugar, quería agradecer a mis dos tutores, Jose Díaz y Nadia Yahlali, el intentar transmitirme en todo momento tantos conocimientos como les ha sido posible, iendo en muchas ocasiones más allá de los conocimientos correspondientes a este trabajo. Por facilitarme cualquier tipo de herramienta o material que agilizase mi proceso de aprendizaje o mi trabajo. Agradecer en concreto que me haya facilitado las medidas de calibración de los PMTs utilizados en el prototipo, además de ciertas macros de ROOT, desarrolladas por ellos mismos, que me sirvieron como base para el desarrollo de los macros que se utilizaron en este trabajo. También por informarme acerca de muchas de las herramientas de las que se dispone en el laboratorio y, en general, de las que dispone el IFIC, algunas de los cuales no eran necesarios para el desarrollo del TFM. Por utilizar gran parte de su tiempo en ayudarme a solucionar muchos de los problemas que me han ido surgiendo en el laboratorio. Finalmente, agradecer el haberme incluirme en todo tipo de procesos referidos al proyecto \textit{Tritium}. 



En segundo lugar agradecer a los ingenieros electrónicos D. Marc Terol, D. Vicente Álvarez  y D. Javier Rodríguez que, a pesar de no disponer de tiempo libre, intentar ayudarme en cualquier duda que tuviese sobre el equipo electrónico o intentar facilitarme material que necesitase, al Dr.  Carlos Acevedo, de la Universidad de Aveiro, Portugal, por ayudarme en el aprendizaje de la programación con Geant 4, facilitarme el programa de simulación con un diseño rectilíneo del prototipo y resolverme cualquier duda acerca de las simulaciones o la programación, y a  D.  José Manuel Monserrate, del Servicio de Mecánica del IFIC, por su ayuda con la fabricación del prototipo.

Finalmente agradecer a los investigadores del IFIMED que nos prestasen su tiempo e instalaciones para desarrollar la calibración de los SiPM, a Teresa Cámara,  por realizar la disolución radiactiva, el proceso de llenado de ésta en el prototipo y por facilitarme cualquier tipo de material o ayuda que necesitase y  a Rosa Carrasco por supervisar el protocolo de manipulación de la solución radiactiva y aportar parte del material necesario para el llenado del prototipo y su señalización como fuente radiactiva.