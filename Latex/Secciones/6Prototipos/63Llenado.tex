Para realizar el llenado, en primer lugar necesitamos preparar la solución de agua tritiada. Para ello  empleamos una fuente radiactiva de tritio, es decir, una solución de agua tritiada ($HTO$ en $\ce{H_2}O$) de $2.0169 \pm 0.0017\gram$ de peso y una actividad específica de $A=26.8 \pm 0.6~\mega\becquerel/\gram$ con fecha de calibración del 16 de marzo de 2017. Esta fuente tine una actividad  exenta~\cite{IFIC}. La fuente ha sido proporcionada por PTB (Physikalisch-Technische Bundesanstalt, Braunscheweig and Berlin), Alemania, con número de serie 2005-1442, y número de referencía PTB-6.11-285/03.2017, y fecha de referencia de 1 de enero de 2017~\cite{IFIC}
.

La solución que preparamos para el experimento, denominada solución patrón, consiste en diluir la ampolla que contiene la solución anterior en  0.5L de agua hiperpura (destilada 5 veces y sin cationes, es decir conductividad muy baja), la cual ha sido proporcionada por el LARUEX de Cáceres, Universidad de Extremadura. Esta disolución fué realizada por Teresa Cámara en el laboratorio de radiactividad ambiental, LARAM, de la universidad de Valencia.



Finalmente, con la solución patrón ya preparada, se realizó el proceso de llenado en la gammateca del IFIC, sala debídamente acondicionada para manipulación de fuentes radiactivas. Se prepararon un total de $500~\cm^3$ de solución patrón.

El mecanismo seguido para el proceso de llenado se describe a continuación:

\begin{enumerate}
\item{} En primer lugar se dispuso de una bandeja de plástico recubierta con material absorbente en el interior de la cual se realizaría todo el proceso de llenado. El objetivo de esta era evitar en la medida de lo posible una contaminación debido a un desbordamiento imprevisto en el proceso de llenado. 

\item{} En segundo lugar, con ayuda de una pipeta y un embudo de cristal, se procedió a llenar una bureta. Para mayor seguridad se fijo la bureta con un soporte de laboratorio. 

\item{} En tercer lugar, con la bureta ya llena del agua tritiada, se procedió a introducir esta en el prototipo. Para ello se introdujo la punta  de la bureta en el orificio de $8~\milli\meter$ del prototipo anteriormente descrito y, lentamente, se procedió al llenado del mismo. El proceso terminó cuando se introdujeron $39~\cm^3$ ya que, como se ha mencionado anteriormente, este es el límite del prototipo.

\item{} En cuarto lugar se procedió a cerrar los dos orificios del prototipo anteriormente mencionados. Además, el resto de la disolución sobrante se vertió en una botella suficientemente segura, en la cual se conservará para un futuro uso. Esta disolución, junto con el material empleado en el proceso de llenado se introdujo en una bolsa de plástico, que a su vez  se introdujo en una caja de cartón, y todo ello fue debidamente guardado en el LARAM.

\item{} En último lugar se trasladó el prototipo, debidamente rellenado y sellado, al Laboratorio de Reacciones Nucleares del IFIC (025)

\end{enumerate}

