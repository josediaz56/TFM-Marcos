Para llevar a cabo esta caracterización de los SiPM se ha necesitado de la instrumentación que se especifica a continuación:

\begin{enumerate}
\item {} En primer lugar se necesitaba una \textbf{cámara de control de temperatura}. 
\newline
Dado que la cámara existente en el laboratorio del IFIC no estaba configurada, tuvo que utilizarse la cámara que se encontraba en el IFIMED. Esta cámara (marca DYCOMETAL, modelo CCM 81) se presenta en la figura~\ref{sistematemperatura} izquierda.
Este sistema dispone de un panel de control con el que se puede especificar la temperatura y humedad a la que queremos trabajar en el interior de la cámara. Sin embargo, para asegurar el correcto funcionamiento de la cámara, debíamos trabajar siempre en la zona 1 de acuerdo al diagrama de estados  en la ficha técnica de la cámara, mostrado en la figura~\ref{sistematemperatura} derecha. Posee un interior metálico que permite una rápida estabilización ante posibles cambios de las condiciones en su interior y, además, actúa a modo de jaula de Faraday. 

\begin{figure}[htb]
\centering
{
%\subfloat[Espectro de emisión]
\includegraphics[scale=0.3]{InteriorTemperatura.png} 
}
{
%\subfloat[Espectro de emisión]
\includegraphics[scale=0.3]{FichaTecnica.png} 
}
\caption{Sistema de control de temperatura y diagrama de estados~\cite{dycometal}\label{sistematemperatura}}
\end{figure}

La incertidumbre  del punto de trabajo es $0.1~\celsius$ para la temperatura y $0.5\%$ para la humedad relativa. Estas incertidumbres se determinaron observando la oscilación  máxima en la pantalla del panel de ambos parámetros tras llegar a la estabilidad en el interior de la cámara.
En el interior de esta cámara se encontraba nuestra fuente de luz que proporcionaba la señal de entrada del sistema que pretendíamos medir con el SiPM, el propio SiPM, la tarjeta de conversión intensidad-voltaje y cableado que hacía posible la interacción con estos dispositivos. Además, hay que tener en cuenta que esta cámara no actuaba como caja negra, por lo que para conseguir reducir la posible entrada de luz del exterior se taparon todas las posibles entradas de luz del sistema con cinta metálica y, además, se cubrió el dispositivo con una tela negra especial. Con todo esto, conseguimos reducir el fondo del sistema hasta un nivel adecuado.

\item {} En segundo lugar necesitamos instrumentos nos permitiesen alimentar tanto el SiPM como la tarjeta de conversión intensidad-voltaje. Se utilizo  un \textbf{electrómetro} (marca KEITHLEY, modelo 6517B) para alimentar el SiPM y un \textbf{generador de tensión} (marca ISO-TECH, modelo IPS-4303) para alimentar la tarjeta de conversión.

Se utilizo el electrómetro para el SiPM, ya que este dispone de una fuente de tensión de hasta $1000\volt$. Por el contrario, el ISO-TECH sólo dispone de una tensión máxima de $30~\volt$. Poseen una resolución  inferior al milivoltio e inferior a $0.1~\volt$ respectivamente, suficiente para que las ganancias que dependen del voltaje (ganancia del SiPM y de la tarjeta respectivamente) no varíe cuando fijamos el voltaje.

\item {} En tercer lugar, se necesitaba una \textbf{fuente de luz} que simulase la emisión de los fotones de la fibra centelleadora. 
\newline
Se utilizó un \textbf{diodo LED} (de la empresa Roithner Laser technik) que emite fotones de una longitud de onda de $435~\nano\meter$~\cite{datasheetLED}, en la zona del azul. Esta es la longitud de onda que necesitamos en nuestro experimento para calibrar el SiPM, ya que corresponde a la longitud de onda a la cual el espectro de emisión de las fibras centelleadoras BCF-12 tiene su máximo.

%\newpage
\item {} En cuarto lugar, para alimentar este diodo LED, se necesitó un \textbf{generador de pulsos} (marca Agilent, modelo 33250A). Este generador de señales nos permite especificar la forma del pulso que pretendemos proporcionar y sus características. Este tenía que ser capaz de formar un pulso suficientemente estrecho para que el SiPM detectase unos pocos fotones. 

En concreto alimentamos el diodo LED con un pulso cuadrado. Los parámetros que nos permite especificar el generador de señales para un pulso de esta forma son frecuencia (o período), high level (o amplitud), low level, offset, anchura del pulso y tiempo de atenuación. Para nuestro estudio, los valores de estos parámetros que nos daban un mejor resultado desde el punto de vista experimental fueron una frecuencia de $20~\hertz$, high level de $2.275~\volt$, low level de $1~\volt$, offset de $1.638~V_{dc}$, anchura del pulso de $12~\nano\second$ y tiempo de atenuación de $5~\nano\second$.
Este generador de señal nos proporciona una segunda señal, denominada señal de sincronización, la cual podemos utilizar como trigger para determinar el instante de tiempo en el que se activa la señal luminosa.

\item {} En quinto lugar, utilizamos un \textbf{SiPM} para detectar los fotones emitidos por el LED. En concreto, se utilizó el modelo S13360-1375CS de Hamamatsu Photonics, que es un fotomultiplicador cerámico de silicio que presenta una ganancia típica de $G=4 \cdotp 10^6$ y una eficiencia de fotodetección típica del $50\%$ a $25~\celsius$ y $V_{ov}=V_{op}-V_{bd}=3~\volt$. Su campo espectral  es de $[270-900]\nano\meter$~\cite{datasheet SiPM}.
Este SiPM compuesto por un total de $285$ pixels de $75~\micro\meter$ cada uno dando lugar a una superficie total activa de $1.3\times1.3~\milli\meter^2$ frente a su superficie total que es de $6\times5~\milli\meter^2$. Puede verse reflejado en la figura~10 izquierda~\cite{datasheet SiPM}. 

\item {} Hay que tener en cuenta que el SiPM nos proporciona un pulso de intensidad a la salida y  necesitamos convertir este en un pulso de voltaje para, de esta forma, poder introducirla al osciloscopio para realizar el posterior análisis. Para ello emplearemos en sexto lugar una \textbf{tarjeta conversora} de intensidad en voltaje que puede verse reflejada en la figura~\ref{TarjetaSiPM} derecha, en la cual se encuentra el SiPM utilizado conectado a la misma. 

\begin{figure}[htb]
\centering
{
%\subfloat[Espectro de emisión]
\includegraphics[scale=0.55]{SiPM.png} 
}
{
%\subfloat[Espectro de emisión]
\includegraphics[scale=0.23]{tarjeta.png} 
}
\caption{SiPMs (izquierda) y tarjeta de alimentación conversora intensidad-voltaje (derecha)\label{TarjetaSiPM}}
\end{figure}
Esta es una tarjeta desarrollada en el marco del proyecto NEXT.Esta consiste de dos entradas donde podemos conectar dos SiPM distintos. La tarjeta contiene el circuito  de la figura~\ref{esquemacircuitotarjeta} para cada SiPM:

\begin{figure}[hbtp]
\centering
\includegraphics[scale=0.3]{CircuitoTarjeta.png}
\caption{Circuito de una tarjeta tipo~\cite{datasheet SiPM}\label{esquemacircuitotarjeta}}
\end{figure}
\newpage
La tarjeta tiene una ganancia de $G=170$. Este será el valor que utilizaremos en el análisis posterior para calcular la ganancia de los SiPM.

\item {} Seguidamente se observó que, únicamente alimentando la tarjeta, antes de alimentar el SiPM y la led, obteníamos una perturbación externa a nuestro experimento. Esta se presenta en la figura~\ref{Ruido}. 

\begin{figure}[hbtp]
\centering
\includegraphics[scale=0.4]{ruido.png}
\caption{Ruido electrónico\label{Ruido}}
\end{figure}
Se intento caracterizar esta señal de perturbación, ajena a nuestro experimento y, a priori, de origen desconocido. Tenía una amplitud máxima de $4~\milli\volt$ y una frecuencia bastante irregular entorno al $\mega\hertz$. Finalmente se vio que era debido por una parte a la emisión producida por la antenas de Radiotelevisión Valenciana y, por otra parte, de las distintos intrumentos electrónicos conectados a la red eléctrica del IFIC, ya que esta posee una toma a tierra bastante irregular. 
Con la presencia de este ruido no podíamos realizar medidas ya que introducía una componente de ruido tal que estropeaba por completo el resultado de la medida. Con el fin de solucionar este problema se dispuso de un \textbf{filtro pasa banda} (GUARD LCD2 650 AP) para eliminar este ruido.

\item {} Finalmente, como se ha mencionado anteriormente, necesitamos una \textbf{manta negra} especial que apantallase la entrada de fotones del exterior, ya que el sistema de control de temperatura no actuaba como caja negra.

Una alternativa podría haber sido introducir una caja negra en el interior del sistema de control de temperatura y introducir el diodo Led, el SiPM y la tarjeta en su interior. Sin embargo de esta forma no conseguimos un control de la temperatura y humedad en su interior.

\end{enumerate}
