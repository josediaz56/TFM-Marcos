%\section {Bibliografía}
\begin{thebibliography}{99}
\bibitem{Ivo} \textsc{Ivo van Vulpen} y \textsc{Ivan Angelozzi},
\textit{The Standard Model Higgs Boson}
\bibitem{JI} \textsc{José Ignacio Illana},
\textit{EL MODELO ESTÁNDAR Y SU FENOMENOLOGÍA}
\bibitem{notas} \textsc{Siannah Peñaranda Rivas},
\textit{"Quantum Field Theory" Photons and the electromagnetic field}
\bibitem{resumen} \textsc{M. Gomez-Bock}, \textsc{M. Mondragón}, \textsc{M. Mühlleitner}, \textsc{R. Noriega-Papaqui}, \textsc{I. Pedraza}, \textsc{M. Spira} y \textsc{P.M. Zerwas},
\textit{Rompimiento de la simetría electrodébil y la física del Higgs: Conceptos Básicos}
\bibitem{Martin} \textsc{Francis Halzen} y \textsc{Alan D. Martin},
\textit{QUARKS AND LEPTONS: An Introductory Course in Modern Particle Physics}
\bibitem{PDG} \textit{Partícle data group} (http://pdg.lbl.gov/)
\bibitem{QCD} \textsc{Matthias Jamin},
\textit{QCD and Renormalisation Group Methods}
\bibitem{Rojo}\textsc{Juan Rojo},
\textit{The Strong Interaction and LHC phenomenology}
\bibitem{Thomson} \textsc{Mark Thomson},
\textit{Particle Physics}
\bibitem{notas1} \textsc{Mark Thomson},
\textit{Particle Physics} Handout 1: Introducción
\bibitem{notas2} \textsc{Mark Thomson},
\textit{Particle Physics} Handout 1: Symmetries and Quark Model
\bibitem{Santos} \textsc{Joaquín Santos Blasco},
\textit{UNITARY ANALYSIS OF THE SCALAR SECTOR OF THE STANDARD MODEL}
\bibitem{string} \textsc{J. Marsano},
\textit{String Thery in the LHC Era} Lecture 3: Why do we need the Higgs? 

\end{thebibliography}