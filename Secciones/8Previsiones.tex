Cabe mencionar que se utilizaron estos SiPM de forma provisional ya que son los que estaban disponibles en el laboratorio y son de tipo CS, es decir, disponían de dos terminaciones para una conexión rápida y no permanente. Sin embargo, la propuesta final de tritium sería utilizar fotomultiplicadores modelo S13360-6075PE,  que presentan exactamente las mismas características que los anteriormente mencionados pero con un mayor tamaño, (6x6) $mm^2$ de la superficie total activa, y, por extensión, mayor número de pixels (6000). Este mayor número de pixels repercute en un rango espectral del SiPM distinto [320-900] nm. Además estos SiPM son de tipo PE, es decir, presentan terminaciones que tiene que ir soldadas a la placa por lo que, para su utilización, se necesita disponer de la placa final que se utilizará en el prototipo. Estos pueden apreciarse en la figura cinco central.



Hay que tener en cuenta que esta tarjeta únicamente nos servirá para las primeras pruebas del prototipo. 

En futuras prueba sobre el prototipo se utilizará una tarjeta que se utilizó para el prototipo del proyecto NEXT, proyecto que requería unas condiciones muy similares al nuestro. Esta tarjeta permite conectar 4 SiPM y permite el tratamiento de forma óptima de estas cuatre señales a partir de multiplexadores y reles (que eligen una señal u otra sin que aparezca ruido inducido entre señales) o amplificadores operacionales que dan la opción de amplificar la señal en caso de ser necesario. Lo más importante de esta tarjeta es que esta proparada para realizar un proceso de automatización, es decir, calibra los SiPM de forma automática y nos permite interaccionar con todas las partes importantes de nuestro experimento, tales como fuente de tensión de alimentación de distintos componentes del sistema, osciloscopios, etc, a partir de un programa llamado LabView.

Hay que tener en cuenta que estas tarjetas de conversión solo valdrán para el prototipo ya que este solo contendrá un bunch de fibras (pack de 35 fibras). Que pueden ser leidas únicamente por un SiPM. El diseño final del prototipo contendrá un número mucho mayor de bunch de fibras centelleadoras y, por extensión, necesitará un número mayor de SiPM para leer todas estas. Es en este punto cuando cobra una vital importancia el hecho de la automatización tanto en calibración de SiPM como en control de componentes ya que calibrar un SiPM es un trabajo costoso pero abordable pero calibrar un número elevado de SiPM (en principio se han previsto 64 SiPM) es un trabajo que llevaría demasiado tiempo. Además, para poder controlar de forma efectiva un número tan elevado de SiPM es necesario un control automático. 


Desarrollar cámara temperatura (referencia)