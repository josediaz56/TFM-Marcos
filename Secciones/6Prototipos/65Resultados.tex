Ahora, dado que ya disponemos de todas las partes del sistema convenientemente dispuestas y calibradas, procedemos a realizar una primera medida.

Hay que tener en cuenta que en estas primeras medidas no necesitamos un sistema de control de temperatura ya que vamos a utilizar PMT y estos no son tan sensibles a la temperatura como los SiPM. Únicamente deberemos ser capaces de evitar oscilaciones grandes de temperatura ($\Delta T=15-20ºC$), los cuales si afectan de forma apreciable a la ganancia del PMT. Esto lo podemos conseguir manteniendo el aire acondicionado de la sala experimental encendido y programado a una misma temperatura durante las medidas. 

Sin embargo, cuando vayamos a realizar las medidas del prototipo con SiPMs, si necesitaremos este control exaustivo de la temperatura ya que, como se vió en la sección de calibrado del SiPM, \ref{sec:Temperatura}, existe una dependencia muy marcada. Se necesitará adquirir un sistema de control de temperatura o, en su defecto, como construir y calibrar uno nosotros mismos.

La primera medida se realizará sin coincidencia, pasando directamente la señal 1 al MCA. El MCA nos permite obtener un histograma energético.

Se obtendrá un primer histograma del prototipo con agua hiperpura y sin tritio. Dado que el agua hiperpura idealmente no añade ningun tipo de contribución al histograma llamaremos a esta medida señal de fondo. 

Seguidamente se rellenará el prototipo con agua tritiada y se obtendrá un segundo histograma. En esta segunda medida se pude observar en el display del MCA que el número de cuentas medidas por segundo era mayor que el la medida del fondo. Esto es un indicativo de que estamos detectando tritio. 

Finalmente se normalizan cada uno de los histogramas al tiempo correspondiente a su medida para obtener histogramas de actividad en lugar de un histograma de sucesos. Los tiempos asociados al histograma del tritio y del fondo son $T_S=154143~\second$  y $T_B=246987~\second$, respectivamente, medidas largas para tener suficiente estadística. El resultado puede verse en la figura 26 izquierda. Además se ha añadido una segunda imagen, figura 26 derecha, en la cual se ha realizado representado la señal con tritio a la cual se le ha extraido la señal de fondo, es decir, en esta segunda imagen encontramos la señal debida únicamente al tritio.

\begin{figure}[hbtp]
\centering
\includegraphics[scale=0.4]{Tritium3-fcpst2.png}
\caption{Histograma energético de la señal del prototipo de tritios y del fondo superpuestos.\label{tritiofondo}}
\end{figure}

En esta figura podemos observar que estamos detectando un pico de señal de aproximadamente $0.045~\becquerel$ y una señal debida únicamente a tritio de aproximadamente $0.01~\becquerel$. POdemos apreciar la existencia de una zona entre los canales 450 y 700 donde el fondo supera a la señal. Habrá que indagar más en este prototipo para encontrar la explicación de este fenómeno.

Para ver hasta que punto esta funcionando nuestro sistema de forma adecuada podemos realizar un rápido cálculo para ver cual es la actividad detectada que deberíamos esperar. Para ello tenemos en cuenta que:
\begin{itemize}
 \item{} Por un lado cada fibra posee un radio de $0.5~\mm,$ una longitud efectiva de $20~\cm$, donde se ha tenido en cuenta que los aros metálicos de los extremos del bunch y la parte de las fibras que sobresalen del prototipo para acoplar a los PMTs no contribuyen a la señal de detección del tritio. Además suponemos que la profundidad del agua tritiada que esta contribuyendo a la señal es de $5~\mu m$ desde la superficie de la fibra (recorrido libre medio de los electrones del tritio en el agua, sec. $\ref{sec:Introduccion}$). Con todo esto podemos ver que el volumen efectivo del agua tritiada que esta contribuyendo a la señal debida a cada fibra es  $\pi \cdotp 10^{-6}~l \approx 3.1415 10^{-6}~l$. 
 
Por tanto, teniendo en cuenta que el prototipo únicamente dispone de un bunch formado por 35 fibras centelleadoras el volumen efectivo final del agua tritiado que estamos detectando será el anterior multiplicado por 35, es decir, $1.09956\cdotp 10^{-4}$ L. 

Hay que tener en cuenta que en esta última multiplicación se ha supuesto que el volumen del agua tritiada asociadas a cada fibra forman un conjunto disjunto y sabemos que esto no es así ya que se producen solapamientos entre ellos. Debido a ello este cálculo será únicamente indicativo.

\item{} Por otro lado, teniendo en cuenta que la disolución posee $2.0169 \pm 0.0017~\gram$ de tritio con una actividad específica de $26.8 \pm 0.6~\mega\becquerel/\gram$ disueltos en medio litro de agua hiperpura (Sec. $\ref{sec:Llenado}$) podemos calclularnos la actividad total de la disolución, la cual será aproximadamente de $108.11~\mega\becquerel/L$. 

Por tanto, si en medio litro hay aproximadamente $1.0811\cdotp 10^{8}$ desintegraciones por segundo en el volumen calculado anteriormente habrá aproximadamente $2.3774\cdotp 10^{4}$ desintegraciones por segundo.

Finalmente, teniendo en cuenta que la eficiencia de la fibras y de los PMTs son, aproximadamente, 3\% y 30\% respectivamente y suponiendo que la cadena electrónica posee una eficiencia del 100\% llegamos a que la actividad que deberíamos detectar es, aproximadamente $213.97~\becquerel$.

\end{itemize}

Podemos ver que estamos detectando 4 ordenes de magnitud menos de lo que deberíamos. Por un lado esto es debido a imperfecciones del sistema pero la principal fuente de la pérdida de la señal es el hecho de que las fibras empleadas en este prototipo no poseen clad. 

El clad hace que los fotones sean conducidos por el interior de las fibras a partir de reflexiones hasta el contador de fotones, es decir, actua como guia de luz, por tanto, en nuestro prototipo, muchos de los fotones de emisión de las fibras escaparán de estas llegando al agua y, por tanto, producirán pérdida de señal. 

Como resultado únicamente detectaremos el porcentaje asociado al ángulo sólido cubierto por las fibras centelleadoras respecto al total de electrones emitidos por el tritio, cuya emisión supondremos isótropa. Además, del total de fotones reemitidos por las fibras centelleadoras (de nuevo se supone emisión isótropa) ante la detección de este electrón, únicamente detectaremos el porcentaje asociado al ángulo sólido cubierto por la cara final de la fibra centelleadora. También hay que tener en cuenta que el porcentaje de fibra que se encuentra en la parte inferior de la U que conforma el prototipo apenas intervendrá en la señal ya que vemos que prácticamente la totalidad de esta será perdida (su ángulo sólido es cero). 

Podemos realizar una rápida estimación de la magnitud relativa de estos ángulos sólidos para ver su importancia en la pérdida de la señal. Integrando la expresión del ángulo sólido esta toma la siguiente forma: $\Omega=2\pi(1-\cos{\theta})$ donde $\theta$ es el ángulo que subtiene la superficie que queremos calcular$~\cite{unizar}$. Por tanto, dado que la superficie total de emisión será $4\pi$ ($\theta=180º$) el factor de reducción debido al ángulo sólido será $\frac{\Omega}{4\pi}=\frac{1-\cos{\theta}}{2}$. 

Si nos calculamos el ángulo y, por extensión, este factor vemos que, para el caso más extremo, el factor debido al ángulo sólido subtendido por la fibra centelleadoras es aproximadamente $0.5$, es decir, aproximadamente la mitad de los electrones que se producen en este punto del agua tritiada pasan por la fibra. El resto de puntos del volumen efectivo poseerán un factor igual o mayor. Sin embargo, la reemisión de fotones de las fibras al detectar un electrón posee un factor realmente pequeño. Por ejemplo, un punto situado a 2 cm de la cara final de la fibra posee un factor de $3.12\cdotp 10^{-4}$ llegando a valer $0$ para los puntos correspondientes a la parte inferior del prototipo como se ha mencionado anteriormente. Vemos por tanto que si llega a ser un factor realmente relevante y que podría explicar esta gran pérdida de señal. 

Por tanto, debida a la necesidad de recolectar el máximo de luz, un paso inmediato en el siguiente prototipo será incluir fibras con clad que nos permitan recolectar un mayor porcentaje de luz. El problema reside en que el grosor actual del single clad comercial de las fibras Saint-Gobain es de, aproximadamente, $4\%$ del diametro, es decir, $40~\mu m$. Prácticamente ningun electrón conseguirá pasar el clad y ser detectado en el núcleo de la fibra por lo que habrá que considerar otros mecanismos de guia de ondas. Una posible solución a este problema sería incluir nosotros el clad en las fibras a partir del proceso de vaporación.... Esto lo podemos realizar en el ICMOL, departamento asociado al IFIC, con el cual ya se han realizado trabajos similares anteriores. De esta forma conseguiríamos un clad con un espesor del orden de... , espesor suficientemente pequeño para que un porcentaje aceptable de electrones derivados de la desintegración del tritio puedan superarlo.