Para realizar el llenado en primer lugar necesitamos preparar la solución de agua tritiada. Para ello disponemos de una fuente radiactiva de tritio, es decir, una solución de agua tritiada ($HTO$ en $\ce{H_2}O$) de $2.0169 \pm 0.0017\gram$ de peso y una actividad específica de $A=26.8 \pm 0.6~\mega\becquerel/\gram$ con fecha de calibración del 16 de marzo de 2017. Esta fuente presenta una actividad tal que puede considerarse exenta.

Esta ha sido proporcionada por PTB (Physikalisch-Technische Bundesanstalt, Braunscheweig and Berlin), Alemania. Presenta un número de serie 2005-1442, un número de referencía PTB-6.11-285/03.2017, una fecha de referencia de 1 de enero de 2017.

La solución que prepararemos para el experimento, denominada solución patrón, consiste en diluir la ampolla que contiene la solución anterior con 0.5L de agua hiperpura, la cual ha sido proporcionada por el LAREUX de Cáceres, Universidad de Extremadura. Esta disolución fué realizada por Teresa Cámara en el laboratorio de radiactividad ambiental, LARAM, de la universidad de Valencia.

El agua hiperpura consiste...

Finalmente, con la solución patrón ya preparada, se realizó el proceso de llenado en la gammateca del IFIC, sala debídamente acondicionada para manipulacón de fuentes radiactivas. Se prepararon un total de $50~\cm^3$ de solución patrón.

El mecanismo seguido para el proceso de llenado se describe a continuación:

\begin{enumerate}
\item{} En primer lugar se dispuso de una bandeza de plastico recubierta con material absorvente en el interior de la cual se realizaría todo el proceso de llenado. El objetivo de esta fue evitar en la medida de lo posible una contaminación debido a un desbordamiento en el proceso de llenado. 

\item{} En segundo lugar, con ayuda de una pipeta y un embudo de crital, se procede a llenar una bureta. Para mayor seguridad se fijo la bureta con un soporte de laboratorio. 

\item{} En tercer lugar, con la bureta ya llena del agua tritiada, se procede a introducir esta en el prototipo. Para ello se introduce la punta final de la bureta en el orificio de 8 mm del prototipo anteriormente descrito y, lentamente, se procede al llenado del mismo. El proceso terminará en el momento en que se hayan introducido $39~\cm^3$ ya que, como se ha mencionado anteriormente, este es el límite del prototipo.

\item{} En cuarto lugar se procederá a cerrar los dos orificios del prototipo anteriormente mencionados. Además, el resto de la disolución sobrante se verterá en una botella suficientemente segura en la cual se conservará para un futuro uso. Esta disolución junto con el material empleado en el proceso de llenado se introducirá en una bolsa de plastico, la cual se introducirá en una caja de carton, y todo ello será debidamente guardado en el LARAM.

\item{} En último lugar se trasladará el prototipo, debidamente rellenado y sellado, al laboratorio de reaciones nucleares del IFIC (025)

\end{enumerate}

