Con todo el prototipo dispuesto para medir procedemos a configurar la electŕonica para obtener una señal adecuada. El objetivo será obtener un espectro energético de experimento con ayuda de un analizador multicanal analógico, MCA. 

Para ello necesitamos realizar una seríe de transformaciónes a la señal para que esta pueda ser analizada de manera adecuada por el MCA. El esquema electrónico utilizado se muestra en la figura veinti tres.

IMAGEEEEEEN

A continuación se explican los módulos y el camino seguido por la señal para este preparamiento:
\begin{enumerate} 
\item{} En primer lugar sacaremos la señal de cada fotomultiplicador de la caja negra utilizada para apantallar la luz del exterior. Ello lo conseguimos con ayuda de cables BNC  ya que la caja dispone de puertos BNC macho para sacar la señal. 

\item{} Hay que tener en cuenta que los PMTs ofrecen señales analógicas y negativas. Por tanto, en segundo lugar necesitamos convertir estas señales en señales lógicas de estándar NIM para poder ser tratadas con tecnología NIM. Esto lo conseguimos con ayuda de un módulo discriminador 

\item{} Además esta señal será muy débil ya que ya 

\end{enumerate}



